\documentclass[english, 11pt]{article}
\usepackage{CourseNotes}
\usepackage{amssymb}
\usepackage{systeme}
\usepackage{esint} %for closed integral symbols
%\usepackage{titlesec}

% Uncomment these for a different family of fonts
% \usepackage{cmbright}
% \renewcommand{\sfdefault}{cmss}
% \renewcommand{\familydefault}{\sfdefault}

\newcommand{\thiscoursename}{Firedrake Notes}
\newcommand{\thisprof}{Supervisors: Dr Pawar, Prof Piggott, Dr Sebille}
\newcommand{\me}{Alexander Kier Christensen}
\newcommand{\thisterm}{2017-18}

\newcommand\given[1][]{\:#1\vert\:}


% Headers
\lhead{\thisterm}


%%%%%% TITLE %%%%%%
\newcommand{\notefront} {
\pagenumbering{roman}
\begin{center}

\textbf{\Huge{\noun{\thiscoursename}}}{\Huge \par} \vspace{0.1in}

\vspace{0.1in}

  {\noun \thisprof} \ $\bullet$ \ {\noun \thisterm} \ $\bullet$ \ {\noun {Imperial College London}} \\

  \end{center}
  }

% Begin Document
\begin{document}


  % Notes front
  \notefront
  % Table of Contents and List of Figures
  \tocandfigures
  % Abstract
  %\doabstract{These notes are intended as a resource for myself; past, present, or future students of my thesis, and anyone interested in the material. The goal is to provide an end-to-end resource that covers all material discussed in the thesis displayed in an organized manner. If you spot any errors or would like to contribute, please contact me directly.}

 % \begin{defn}[addition]\label{addition}
 % Two {\bf addition} operation adds two numbers, for $a, b \in \R$, their sum is
 % \[ a + b \]
 % \end{defn}

  %The \nameref{addition} rule is very good.

  %\begin{lstlisting}[language=lisp]
  %(define sum (lambda args (foldr + 0 args)))
  %\end{lstlisting}

  %\tc{this is code}

  %%%%%%%%%%%%%%%%%%%%%%%%%%%%%%%%%%%%%%%%%%%%%%%
\setcounter{section}{1}
\begin{center}
\section{Key Notes to NEVER FORGET}
\end{center} 
  
  \subsection{"Integration by Parts"}
  
  This comes up constantly in FEM stuff, when expressing problems in variational form. The usual spiel is to say "multiply by a test function $v$ and then integrate by parts", to obtain the desired form. This hides a number of key subtle steps that otherwise look like magic.\\
  \\
  First to note is that "integrate by parts" really means "apply (a corollary of) the Divergence Theorem":
  
  \begin{thm}[Divergence Theorem]
  If $\Omega$ is a compact subset of $\R^N$ with a piecewise smooth boundary $\partial\Omega = \Gamma$, and if $\bm{F}$ is a continuously differentiable vevctor field defined on a neighbourhood of $\Omega$ then we have:
  
  \[ \int_{\Omega} (\nabla \cdot \bm{F}) \ d\Omega = \oiint \limits_{\Gamma} (\bm{F} \cdot \bm{n}) \ d\Gamma \]
  
  where $\bm{n}$ is the outward pointing unit
  normal field of the boundary $\Gamma$.  
  \end{thm}
  
  \begin{cor}
  	Replacing $\bm{F}$ with $\bm{F}g$ in the theorem, where $g$ is a scalar function, we get:
  	
  	\[ \int_{\Omega} \bm{F} \cdot (\nabla g) \ d\Omega + \int_{\Omega} g(\nabla \cdot \bm{F}) \ d\Omega = \oiint \limits_{\Gamma} g\bm{F} \cdot \bm{n} \ d\Gamma \]
  \end{cor}
  
  We can apply this corollary to the LHS (i.e. to the terms involving $u$) to rewrite it as the sum of a different volume integral and a surface integral, which can often be made to vanish by applying boundary conditions.
  
  \subsubsection{Example: Linear Poisson Equation}
  
  Let us take an initial easy example of the basic linear Poisson problem:
  
  \begin{align*}
  	(-\Delta u) &= f , \text{ on } \Omega \\
  	u &= 0 , \text{ on } \partial\Omega = \Gamma
  \end{align*}
  
  We multiply both sides by the test function $v$ and integrate to obtain:
  
  \[ \int_{\Omega} (-\Delta u)v \ d\Omega = \int_{\Omega} fv \ d\Omega \]
  
  Now we apply the Corollary to the LHS (replacing $\bm{F}$ with $\nabla u$ and $g$ with $v$) to get: 
  
  \[ \int_{\Omega} \nabla u \cdot \nabla v \ d\Omega + \oiint \limits_{\Gamma} v \nabla u \cdot \bm{n} \ d\Omega = \int_{\Omega} (-\Delta u)v \ d\Omega = \int_{\Omega} fv \ d\Omega  \]
  
  The second term in the new LHS is a \emph{closed} line integral of a grad function and thus equal to the difference of its endpoints, which are the same, hence the term is zero, leaving us with the desired variational form $a(u,v) = L(v)$:
  
  \[ \int_{\Omega} \nabla u \cdot \nabla v \ d\Omega = \int_{\Omega} fv \ d\Omega \]
  
  
  \subsubsection{Example: Nonlinear Poisson Equation}
  
  Let's now look at the following nonlinear Poisson problem: 
  
  \begin{align*}
  	- \nabla \cdot \big( (1 + u) \nabla u \big) &= f , \text{ in } \Omega \\
  	u &= 0 , \text{ on } \partial\Omega = \Gamma
  \end{align*}
  
  We multiply by the test function $v$ and integrate both sides:
  
  \[ \int_{\Omega} \Big( - \nabla \cdot \big( (1 + u) \nabla u \big) \Big) v \ d\Omega = \int_{\Omega} fv \ d\Omega \]
  
  Again we apply the Corollary to the LHS (replacing $\bm{F}$ with $\big( (1 + u) \nabla u \big)$ and $g$ with $v$) to get:
  
  \[ \int_{\Omega} \big( (1 + u) \nabla u \big) \cdot \nabla v \ d\Omega + \oiint \limits_{\Gamma} v \Big( \big( (1 + u) \nabla u \big) \Big) \cdot \bm{n} \ d\Gamma  = \int_{\Omega} \Big( - \nabla \cdot \big( (1 + u) \nabla u \big) \Big) v \ d\Omega = \int_{\Omega} fv \ d\Omega\]
  
  Looking again at the surface integral term on the new LHS, we recall the initial condition $u = 0$ on $\Gamma$ and thus this term simplifies to:
  
  \[ \oiint \limits_{\Gamma} v \big( \nabla u \cdot \bm{n} \big) \ d\Gamma  = 0 \qquad \text{(closed line integral of a grad function)}\]
  
  Leaving us with the desired variational form $F(u;v) = 0$:
  
  \[ \int_{\Omega} \big( (1 + u) \nabla u \big) \cdot \nabla v \ d\Omega = \int_{\Omega} fv \ d\Omega \]
	
	
\newpage	
\section{References}

	\begin{itemize}
		\item \url{https://en.wikipedia.org/wiki/Divergence_theorem}
		\item \url{https://en.wikipedia.org/wiki/Surface_integral}
		\item \url{http://mathinsight.org/gradient_theorem_line_integrals}
	\end{itemize}
	
  
  \end{document}
