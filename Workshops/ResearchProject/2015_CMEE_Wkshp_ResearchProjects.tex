\documentclass[xcolor=x11names,compress]{beamer}

%% General document %%%%%%%%%%%%%%%%%%%%%%%%%%%%%%%%%%
\usepackage{graphicx}
\graphicspath{{graphics/}}
\usepackage[latin1]{inputenc}
\usepackage{times}
\usepackage{verbatim}

\usepackage{tikz}
\usetikzlibrary{arrows,shapes}

\usepackage{color,colortbl}
\usepackage{framed}
\usepackage{textcomp, setspace} %Needed for customization of ``listings''
% package
\usepackage[procnames]{listings} % to display code; don't forget [fragile]
% option after \begin{frame}
\input{inputs/rgb}
\definecolor{shadecolor}{rgb}{1,.9,.3}

\usepackage [autostyle]{csquotes}
\MakeOuterQuote{"}

\lstset{
    backgroundcolor=\color{shadecolor},
    tabsize=4,
    rulecolor=,
    language=python,
        basicstyle=\ttfamily\setstretch{1},
        upquote=true,
        aboveskip={1.5\baselineskip},
        columns=fixed,
        showstringspaces=false,
        extendedchars=true,
        breaklines=true,
        prebreak = \raisebox{0ex}[0ex][0ex]{\ensuremath{\hookleftarrow}},
        frame=single,
        showtabs=false,
        showspaces=false,
        showstringspaces=false,
        identifierstyle=\ttfamily,
        keywordstyle=\color[rgb]{0,0,1},
        commentstyle=\color[rgb]{0.133,0.545,0.133},
        stringstyle=\color[rgb]{0.627,0.126,0.941},
        numbers=left, 
        numberstyle=\tiny, 
        stepnumber=2, 
        numbersep=5pt
}

%%%%%%%%%%%%%%%%%%%%%%%%%%%%%%%%%%%%%%%%%%%%%%%%%%%%%%

%% Beamer Layout %%%%%%%%%%%%%%%%%%%%%%%%%%%%%%%%%%
\usetheme{Madrid}
\usecolortheme{crane}
\useoutertheme[subsection=false,shadow]{miniframes}
\useinnertheme{default}
% \usefonttheme{serif}
\usepackage{palatino}

\setbeamerfont{title like}{shape=\scshape}
\setbeamerfont{frametitle}{shape=\scshape, series = \bfseries}
\setbeamertemplate{frametitle}[default][center]
\setbeamertemplate{headline}{} %suppress headline (navigation pane)

\setbeamertemplate{footline}
{
  \leavevmode%
  \hbox{%
%   \begin{beamercolorbox}[wd=.4\paperwidth,ht=2.25ex,dp=1ex,center]{author in 
% head/foot}%
%     \usebeamerfont{author in head/foot}\insertshortauthor
%   \end{beamercolorbox}%
  \begin{beamercolorbox}[wd=.93\paperwidth,ht=2.25ex,dp=1ex,left]{title in 
head/foot}%
    \usebeamerfont{title in head/foot}\insertshorttitle\hspace*{3em}
  \end{beamercolorbox}}%
  \begin{beamercolorbox}[wd=.07\paperwidth,ht=2.25ex,dp=1ex,center]{}
     \insertframenumber{} / \inserttotalframenumber\hspace*{1ex}
%       \insertframenumber{}
  \end{beamercolorbox}

}
% \setbeamercolor*{lower separation line head}{bg=DeepSkyBlue4} 
% \setbeamercolor*{normal text}{fg=black,bg=white} 
% \setbeamercolor*{alerted text}{fg=red} 
% \setbeamercolor*{example text}{fg=black} 
% \setbeamercolor*{structure}{fg=black}
%  
% \setbeamercolor*{palette tertiary}{fg=black,bg=black!10} 
% \setbeamercolor*{palette quaternary}{fg=black,bg=black!10} 

\renewcommand{\(}{\begin{columns}}
\renewcommand{\)}{\end{columns}}
\newcommand{\<}[1]{\begin{column}{#1}}
\renewcommand{\>}{\end{column}}

\def\signed #1{{\leavevmode\unskip\nobreak\hfil\penalty50\hskip2em
  \hbox{}\nobreak\hfil(#1)%
  \parfillskip=0pt \finalhyphendemerits=0 \endgraf}}

\newsavebox\mybox
\newenvironment{aquote}[1]
  {\savebox\mybox{#1}\begin{quote}}
  {\signed{\usebox\mybox}\end{quote}}

\title{CMEE Workshop: Research projects}
\author{Samraat Pawar}
  
%%%%%%%%%%%%%%%%%%%%%%%%%%%%%%%%%%%%%%%%%%%%%%%%%%%%

\begin{document}

%%%%%%%%%%%%%%%%%%%%%%%%%%%%%%%%%%%%%%%%%%%%%%%%%%%%%%
\begin{frame}[plain]

\title{Workshop: Choosing, designing and executing a dissertation research project}
\vspace{12pt}
\subtitle{MSc/MRes CMEE 2015-16}
\author{
    Samraat Pawar\\
    \vspace{20pt}
  \centering
  \includegraphics[height = .3in]{Imperial_Color1.pdf}
}
 
\titlepage
\end{frame}

%%%%%%%%%%%%%%%%%%%%%%%%%%%%%%%%%%%%%%%%%%%%%%%%%%%%%%
\begin{frame}{What do you want your MSc/MRes Dissertation research to be?}

  \begin{itemize}[<+->] \itemsep8pt
    \item A new piece of scientific work
    \item Something that can be published
    \item Something you find interesting
    \item Something that will get you the highest marks
    \item Something that makes you implement new skills
    \item Something that satisfies your supervisor
    \item Something that can get you a PhD position
    \item All of the above!
 \end{itemize}

\vspace*{6pt}

\pause
\centering 

{\it Go to \url{govote.at} and enter code shown (you can use your phone)}

\end{frame}

%%%%%%%%%%%%%%%%%%%%%%%%%%%%%%%%%%%%%%%%%%%%%%%%%%%%%%
\begin{frame}{The conception and birth of your Dissertation}

\tikzstyle{format} = [draw, thick, fill=yellow!40]
% \tikzstyle{medium} = [ellipse, draw, thick, fill=red!20, minimum height=2.5em]

\begin{figure}
\begin{tikzpicture}[node distance=3.3cm, auto, >=latex', thick]
    % Set bounding box first, or the diagram will change position for each frame.
    \path[use as bounding box] (.5,0) rectangle (10,0);
    \path[->]<1-> node[format] (idea) {IDEA};
    \path[->]<2-> node[format, right of = idea] (design) {DESIGN}
                  (idea) edge node {work} (design);
    \path[->]<3-> node[format, right of=design] (implement) {IMPLEMENT}
                  % node[medium, below of=dvi] (workflow) {"WORKFLOW"}
                  (design) edge node {work} (implement);
                        % edge node[swap] {xdvi} (screen);
    \path[->]<4-> node[format, right of=implement] (thesis) {THESIS}
                  % node[medium, below of=ps] (print) {printer}
                  (implement) edge node {work} (thesis);
                       % edge node[swap] {gs} (screen)
                       % edge (print);
    % \path[->]<5-> (pdf) edge (screen)
                        % edge (print);
    \path[->, draw]<5-> (idea) -- +(0,.8) -| node[near start] {proposal} (design);
    \path[->, draw]<6-> (design) -- +(0,.8) -| node[near start] {workflow} (thesis);
\end{tikzpicture}
\end{figure}

\end{frame}
 
%%%%%%%%%%%%%%%%%%%%%%%%%%%%%%%%%%%%%%%%%%%%%%%%%%%%%%
\begin{frame}{The IDEA phase}

\begin{aquote}{Albert Einstein}
    Anyone who has never made a mistake has never tried anything new
\end{aquote}

  \begin{itemize}[<+->] \itemsep8pt
    \item Split up into three groups and choose one of the assigned heavily cited papers in biology (on bitbucket) -- first come first served!
    \item {\it What made your assigned paper so popular?}
    \begin{itemize} \itemsep2pt
        \item Good timing (was in a popular field)?
        \item A new idea? 
        \item A new method?
        \item General (or specific to an important system)?
        \item Critical --- convincing test of an existing hypothesis?
        \item Integrative --- brought together different disciplines/areas?
        \item The surrounding writing made a good story?
    \end{itemize}
 \end{itemize}

\vspace*{3pt}

\pause
\centering

{\it Go to \url{govote.at} and enter code shown (you can use your phone)} \\
\pause --- Name you paper please -- you have 140 characters!

\end{frame}

%%%%%%%%%%%%%%%%%%%%%%%%%%%%%%%%%%%%%%%%%%%%%%%%%%%%%%
\begin{frame}{The IDEA phase}

  \begin{itemize}[<+->]\itemsep12pt
		\item At Silwood you have three main options to choose a research 
		project (you can exert your creativity in all three!):
    \vspace{6pt}
    \begin{enumerate} \setlength{\itemindent}{-1em}\itemsep6pt
        \small
        \item Come up with something and approach a potential supervisor
        \item Go through supervisor interests and propose an idea 
        \item Choose an advertised project and modify as needed
    \end{enumerate}
    \item See Loehle 1990 (on bitbucket) for some thoughts on creativity
    \item {\it What's your plan?} (Choose one of the three options above)
 \end{itemize}

\vspace*{16pt}

\pause
\centering 

{\it Go to \url{govote.at} and enter code shown (you can use your phone)}

\end{frame}

%%%%%%%%%%%%%%%%%%%%%%%%%%%%%%%%%%%%%%%%%%%%%%%%%%%%%%
\begin{frame}{The IDEA phase}

  \begin{itemize}[<+->]\itemsep20pt
    \item What's your view?
    \item And my view is...
 \end{itemize}

\end{frame}

%%%%%%%%%%%%%%%%%%%%%%%%%%%%%%%%%%%%%%%%%%%%%%%%%%%%%%
\begin{frame}{The IDEA phase}

  \begin{itemize}\itemsep20pt
    \item Supervisor interests: \url{http://goo.gl/okOvZS} (but you can look elsewhere!)
    \item Advertised projects: \url{http://goo.gl/awH7Vf} (but you can look elsewhere!) 
    \item You must have an internal (Imperial College) supervisor ({\it Why?}) 
 \end{itemize}

\end{frame}

%%%%%%%%%%%%%%%%%%%%%%%%%%%%%%%%%%%%%%%%%%%%%%%%%%%%%%
\begin{frame}{The DESIGN phase}

  \begin{itemize}[<+->]\itemsep12pt
    \item Feasibility is important -- the "Medawar zone" \\
    \url{https://en.wikipedia.org/wiki/Medawar_zone}
    \item Where does each of the four papers you discussed lie in the Medawar zone?
    \item Which of the following components would worry you the most in terms of project feasibility?
    \begin{enumerate} \setlength{\itemindent}{-1em}\itemsep6pt
        % \small
        \item Fieldwork
        \item Laboratory experiments
        \item Developing and implementing statistical analyses -- fitting to a model
        \item Developing and analyzing mathematical model
        \item Writing up
        \item Something else
    \end{enumerate}
 \end{itemize}
 
\vspace*{3pt}

\pause
\centering 

{\it Go to \url{govote.at} and enter code shown (you can use your phone)}
\end{frame}

%%%%%%%%%%%%%%%%%%%%%%%%%%%%%%%%%%%%%%%%%%%%%%%%%%%%%%
\begin{frame}{The DESIGN phase}

  \begin{itemize}[<+->]\itemsep16pt
    \item Make your project modular
    \item Have a plan B (and C, if possible) -- what if your primary objective is a dead end? 
    \item Have distinct hypotheses -- thes can be plan B and C
    \item Use a Gantt chart (required in your project proposal)
    \item Get feedback from supervisor 
 \end{itemize}
\end{frame}

%%%%%%%%%%%%%%%%%%%%%%%%%%%%%%%%%%%%%%%%%%%%%%%%%%%%%%
\begin{frame}{The DESIGN phase}

    \centering 
A Gantt chart example
    
    \vspace{10pt}
 \includegraphics[width=\textwidth]{Timeline.pdf}

\end{frame}

%%%%%%%%%%%%%%%%%%%%%%%%%%%%%%%%%%%%%%%%%%%%%%%%%%%%%%
\begin{frame}{The IMPLEMENTATION phase}

  \begin{itemize}[<+->]\itemsep6pt
    \item Make your code modular
    \item Keep your dissertation under version control -- organized under {\tt code}, {\tt data}, {\tt results}, {\tt thesis}... 
    \item Make your analyses reproducible -- you should be able to rerun at touch of a button
    \item Benchmark (perhaps using synthetic data) to estimate how long simulations/analyses are likely to take
    \item Unit test! 
    \item Keep results from all analysis/simulation runs, tag them with a code version number
    \item Get feedback from supervisor at key junctures
    \item Change project design iteratively
    \item Revisit your Gantt chart
 \end{itemize}

\end{frame}

%%%%%%%%%%%%%%%%%%%%%%%%%%%%%%%%%%%%%%%%%%%%%%%%%%%%%%
\begin{frame}{The IMPLEMENTATION phase}

  \begin{itemize}\itemsep6pt
    \item Explicit is better than implicit.
    \item Simple is better than complex.
    \item Complex is better than complicated.
    \item Readability counts.
    \item Special cases aren't special enough to break the rules.
    \item Although practicality beats purity.
    \item Errors should never pass silently.
    \item In the face of ambiguity, refuse the temptation to guess.
    \item There should be one-- and preferably only one --obvious way to do it.
    \item If the implementation is hard to explain, it's a bad idea.
 \end{itemize}

\end{frame}

%%%%%%%%%%%%%%%%%%%%%%%%%%%%%%%%%%%%%%%%%%%%%%%%%%%%%%
\begin{frame}{The IMPLEMENTATION phase}

 \centering
 \includegraphics[width=.4\textwidth]{good_code.png}\\
    {\small \url{http://xkcd.com/}}

\end{frame}

%%%%%%%%%%%%%%%%%%%%%%%%%%%%%%%%%%%%%%%%%%%%%%%%%%%%%%
\begin{frame}{The THESIS writing phase}

  \begin{itemize}[<+->]\itemsep20pt
    \item Start early
    \item Get working on the background and intro early -- will make you think and rethink 
    \item Writing skills improve more slowly than technical (coding) skill -- but keep improving for longer
    \item Don't leave everything for the "last month"!
    \item Get feedback from others -- give presentations!
 \end{itemize}
 
\end{frame}

%%%%%%%%%%%%%%%%%%%%%%%%%%%%%%%%%%%%%%%%%%%%%%%%%%%%%%
\end{document}
